\documentclass{article}
\usepackage{url}

\begin{document}

\section*{Course Overview}
This course is intended for students who are interested in cybersecurity and especially for those who are preparing for PRCCDC (the Pacfic Rim Collegiate Cyber Defense Comptetition). Students taking this course are expected to be self-motivated and interested in learning, with the ability to research a given topic and share their knowledge with others. Class this semester will involve hands-on tutorials involving basic hardening, attack, and defense skills on Windows and *Nix based operating systems.

\section*{Course Objective}
Over the semester, students will have an opportunity to practice researching and presenting on assigned topics that are relevant to PRCCDC, with the highly encouraged option to include more information as they find it useful and interesting. This is a student directed course with the purpose of competing well at PRCCDC, but prior knowledge is not necessary as this semester is targeted at raising every student's knowledge to the level at which we hope to compete.

\section*{Grading}

\begin{tabular}{l  l}
Attendance & 60\% \\
Discussion & 10\% \\
Tutorials & 30\% \\\\
\end{tabular}

\noindent Attendance at 80\% of classes or more is required if you are taking this class for two credits. Understanding that students have many priorities, the expectation is that if you make a reasonable effort to attend every class and communicate if you have a conflicting commitment.\\

\noindent Discussion will involve sharing current events at the beginning of class. Good sources for news are:\\\\
\begin{tabular}{l   l}
YCombinator Hacker News & \url{https://news.ycombinator.com/news?p=1} \\
ArsTechnica & \url{https://arstechnica.com/information-technology/} \\
Wired & \url{https://www.wired.com/category/security/} \\
Twitter & (depending on who you follow)
\end{tabular}\\

\noindent Tutorials will be assigned and chosen at the beginning of the semester. A template with specific expectations will be given, which you are expected to follow in order to receive full credit. In the process of preparing for tutorials, you may find the following resources helpful: \\\\
\begin{tabular}{l  l}
Cyberdefense Github & \url{https://github.com/GhostofGoes/ui-prccdc} \\
%Windows Command Line & \url{https://github.com/Awesome-Windows/awesome-windows-command-line}\\
... & ...\\
\end{tabular}


\section*{Topics}
	\begin{itemize}
	\item How to research and find good resources
	\item Presenting effectively
	\item Commonly used acronyms
	\item Nmap and making a network map
	\item Persistence (sticky keys, scheduled tasks)
	\item Windows firewall (basic rules that should \& shouldn't be there, not locking yourself out by denying RDP)
	\item Windows command line
	\item Powershell
	\item Windows task manager and common tasks (detecting an intruder)
	\item Sysinternals (autoruns, process explorer)
	\item Group policy and active directory
	\item Layer 7 firewalls (Palo Alto)
	\item Layer 3 firewalls (VyOS, pfSense)
	\item Bash fundamentals
	\item Bash scripting
	\item *nix variants and how they differ: FreeBSD, Solaris, Linux, etc.
	\item Linux directory structure and contents
	\item Cron
	\item Networking \& the OSI model
	\item IPv4 addressing
	\item NAT rules
	\item ...
	\end{itemize}

\section*{Schedule}
Quarter 1: Introduction, common acronyms and terminology, overview of PRCCDC and motivation for why and how we're preparing, expectations such as presentation templates, grading, goals, how to ask \emph{good} questions and precisely articulate problems, review of previous competitions, resources that are available, and overview of semester schedule\\\\
Quarter 2: Windows fundamentals\\\\
Quarter 3: Linux fundamentals\\\\
Quarter 4: Networks and domains? \\


\end{document}